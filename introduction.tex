%!TEX root = presentazionelancia.tex

\setlength{\parskip}{\baselineskip} 
\section{Introduction}
\begin{comment}
\begin{frame}[t]
\frametitle{Introduction}
    \begin{center}
    	\includegraphics<1>[width=1\textwidth]{figs/background.jpg}
    \end{center}
\end{frame}
\end{comment}

\begin{frame}
\frametitle{Introduction}
\begin{itemize}
\item Google: 24 PB / day

\item Facebook: 10 million photos + 3 billion ``likes" / day

\item Youtube: 800 million visitors / month

\item Twitter: Doubling its size every year
\end{itemize}
\end{frame}

\begin{frame}
\frametitle{Issues}
\begin{itemize}
\item The most significant issue comes from the size of Big Data.

\item The flip side of size is speed.

\item Transfer cost.

\item \textcolor{red}{Dynamic data --- Data Stream}
\end{itemize}
\end{frame}

\begin{frame}
\frametitle{Dynamic Data Stream}

\begin{itemize}
\item[-] Persistent \textbf{Static} Relations: \textbf{Batch-oriented} data processing

\item[-] Transient \textbf{Dynamic} Data Streams: Real-time \textbf{stream} processing
\end{itemize}


\begin{itemize}
\item \textbf{Architecture Level: } add or remove computational nodes based on the current load

\item \textbf{Application Level: } withdraw old results and take new data into account
\end{itemize}
\end{frame}


\begin{frame}
\frametitle{Objective: parallel and continuous processing for Join operation}
\textbf{Join: } a popular and often used operation in the big data area.

\begin{itemize}
\item Data Driven Join : kNN (Data Parallelism)
\item Query Driven Join : Semantic Join on RDF data (Task Parallelism)
\end{itemize} 

\end{frame}