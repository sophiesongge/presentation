%!TEX root = presentazionelancia.tex

\setlength{\parskip}{\baselineskip} 
\section{Introduction}
\begin{frame}[t]
\frametitle{Introduction}
    \begin{center}
    	\includegraphics<1>[width=1\textwidth]{figs/background.jpg}
    \end{center}
\end{frame}

\begin{frame}
\frametitle{Big Data Everywhere}
\begin{itemize}
\item Google: 24 PB / day

\item Facebook: 10 millions photos + 3 billion ``likes" / day

\item Youtube: 800 million visitors / month

\item Twitter: Doubling its size every year
\end{itemize}
\end{frame}

\begin{frame}
\frametitle{Issues}
\begin{itemize}
\item The most significant issue comes from the size of Big Data.

\item The flip side of size is speed.

\item The cost of network communication in transferring data.

\item \textcolor{red}{The dynamic of data --- Data Stream}
\end{itemize}
\end{frame}

\begin{frame}
\frametitle{Dynamic Data Stream}
Persistent Static Relations $\Rightarrow$ Transient Dynamic Data Streams

Batch-oriented data processing $\Rightarrow$ Real-time stream processing

\begin{center}
$\Downarrow$
\end{center}

\textbf{Architecture Level: } possible to add or remove computational nodes based on the current load

\textbf{Application Level: } able to withdraw old results and take new coming data into account

\end{frame}


\begin{frame}
\frametitle{Objective: parallel and continuous processing for Join operation}
\textbf{Join: } a popular and often used operation in the big data area.

\begin{itemize}
\item Data parallelism $\Rightarrow$ Data Driven Join $\Rightarrow$ kNN
\item Task parallelism $\Rightarrow$ Query Driven Join $\Rightarrow$ Semantic Join on RDF data
\end{itemize} 

\end{frame}